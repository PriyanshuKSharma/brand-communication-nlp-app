\chapter{Results and Discussion}

\section{System Demonstration}
To validate the efficacy of \textbf{Brand Intel}, the system was tested by inputting a YouTube video URL into the dashboard. For this demonstration, we selected the video titled \textbf{"Top 20 Most Played PC Games of 2017"} (\texttt{https://www.youtube.com/watch?v=mrKgqtmQOWs}) to simulate a market research scenario. The system successfully fetched and analyzed 500 comments in real-time.

\section{Sentiment Distribution}
The initial sentiment analysis revealed a highly nostalgic and positive reception, with users discussing their favorite titles.

\begin{table}[H]
\centering
\begin{tabular}{|l|c|c|}
\hline
\textbf{Sentiment Label} & \textbf{Count} & \textbf{Percentage} \\
\hline
Positive & 280 & 56\% \\
\hline
Neutral & 150 & 30\% \\
\hline
Negative & 70 & 14\% \\
\hline
\end{tabular}
\caption{Sentiment Distribution for Gaming Trends Analysis}
\label{tab:sentiment}
\end{table}

As shown in Table \ref{tab:sentiment}, 56\% of the comments were positive. Users frequently used words like "classic", "best", and specific game titles like "PUBG" and "CS:GO". Negative sentiment was lower (14\%) and mostly focused on the exclusion of certain games or disagreements with the ranking.

\section{Topic Modeling Results}
The K-Means clustering algorithm ($K=5$) identified the following distinct topics of conversation:

\begin{table}[H]
\centering
\begin{tabular}{|c|l|l|}
\hline
\textbf{Topic ID} & \textbf{Key Terms} & \textbf{Interpretation} \\
\hline
0 & pubg, fortnite, battle, royale, hype & \textbf{Battle Royale Craze} \\
\hline
1 & csgo, valve, skins, russian, aim & \textbf{Counter-Strike Community} \\
\hline
2 & overwatch, blizzard, heroes, team, toxicity & \textbf{Hero Shooters} \\
\hline
3 & gta, rockstar, online, mods, heist & \textbf{Open World / GTA V} \\
\hline
4 & list, missing, dota, league, legends & \textbf{Ranking Disagreements} \\
\hline
\end{tabular}
\caption{Identified Topics and Interpretations}
\label{tab:topics}
\end{table}

\section{Strategic Recommendations}
By cross-referencing sentiment with topics, the system generated the following strategic insights:

\subsection{Pain Points (Negative Sentiment Clusters)}
\begin{itemize}
    \item \textbf{Topic 4 (Ranking Disagreements):} Users expressed frustration when their favorite games (e.g., Dota 2, LoL) were perceived as ranked too low or ignored.
    \item \textbf{Recommendation for Creators:} When creating ranking content, explicitly state the criteria (e.g., "player count" vs "popularity") to manage audience expectations.
\end{itemize}

\subsection{Winning Themes (Positive Sentiment Clusters)}
\begin{itemize}
    \item \textbf{Topic 0 (Battle Royale):} The explosion of PUBG and Fortnite was a dominant positive theme.
    \item \textbf{Topic 1 (CS:GO):} The legacy community for Counter-Strike remains incredibly active and loyal.
    \item \textbf{Recommendation:} Content focusing on the evolution of the Battle Royale genre or "Retro" reviews of 2017 classics would likely perform well with this audience.
\end{itemize}

\section{Performance Analysis}
The system processed 500 comments in approximately 3.2 seconds on a standard consumer laptop (Intel i7, 16GB RAM). The TextBlob sentiment engine, while fast, struggled with sarcasm (e.g., "Great, another expensive brick" was sometimes classified as Positive due to the word "Great"). This highlights a limitation of rule-based systems compared to transformer-based models.
