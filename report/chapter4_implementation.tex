\chapter{Implementation}

\section{Development Environment}
The project was developed using the following environment and tools:
\begin{itemize}
    \item \textbf{Operating System:} Windows 11
    \item \textbf{Programming Language:} Python 3.11
    \item \textbf{IDE:} Visual Studio Code
    \item \textbf{Version Control:} Git \& GitHub
\end{itemize}

\section{Key Modules and Functions}

\subsection{Text Preprocessing}
The `clean\_text` function is critical for normalizing the noisy social media data. It uses regular expressions (`re` module) to strip unwanted characters.

\begin{lstlisting}[language=Python, caption=Text Cleaning Function]
def clean_text(text):
    """
    Cleans the input text by:
    1. Lowercasing
    2. Removing URLs
    3. Removing mentions (@user) and hashtags (#tag)
    4. Removing special characters and numbers
    """
    if not isinstance(text, str):
        return ""
    
    text = text.lower()
    # Remove URLs
    text = re.sub(r'http\S+|www\S+|https\S+', '', text, flags=re.MULTILINE)
    # Remove mentions and hashtags
    text = re.sub(r'\@\w+|\#\w+', '', text)
    # Remove punctuation and numbers
    text = re.sub(r'[^\w\s]', '', text)
    text = re.sub(r'\d+', '', text)
    
    return text.strip()
\end{lstlisting}

\subsection{Sentiment Analysis}
Sentiment analysis is performed using the `TextBlob` library. The function returns both a polarity score and a categorical label.

\begin{lstlisting}[language=Python, caption=Sentiment Analysis Function]
def get_sentiment(text):
    """
    Returns a tuple: (polarity_score, label)
    Polarity is a float between -1.0 and 1.0.
    """
    blob = TextBlob(text)
    polarity = blob.sentiment.polarity
    
    if polarity > 0.1:
        return polarity, "Positive"
    elif polarity < -0.1:
        return polarity, "Negative"
    else:
        return polarity, "Neutral"
\end{lstlisting}

\subsection{Topic Modeling}
The topic modeling pipeline uses `TfidfVectorizer` to convert text to numbers and `KMeans` to cluster them.

\begin{lstlisting}[language=Python, caption=Topic Modeling Pipeline]
def perform_topic_modeling(texts, n_topics=5):
    # 1. Vectorization
    vectorizer = TfidfVectorizer(max_features=1000, stop_words='english')
    tfidf_matrix = vectorizer.fit_transform(texts)
    
    # 2. Clustering
    kmeans = KMeans(n_clusters=n_topics, random_state=42)
    kmeans.fit(tfidf_matrix)
    
    # 3. Extract Keywords
    feature_names = vectorizer.get_feature_names_out()
    topic_keywords = []
    
    for topic_idx, topic in enumerate(kmeans.cluster_centers_):
        top_features_ind = topic.argsort()[:-6:-1]
        keywords = [feature_names[i] for i in top_features_ind]
        topic_keywords.append(", ".join(keywords))
        
    return kmeans.labels_, topic_keywords, kmeans
\end{lstlisting}

\section{YouTube API Integration}
Fetching data from YouTube requires handling pagination to retrieve more than the default 20 comments. The application implements a `while` loop that checks for the `nextPageToken` in the API response.

\begin{lstlisting}[language=Python, caption=YouTube Data Fetching Logic]
request = youtube.commentThreads().list(
    part="snippet",
    videoId=video_id,
    maxResults=100,
    textFormat="plainText"
)

while request and len(comments_data) < 500: 
    response = request.execute()
    # ... process items ...
    if "nextPageToken" in response:
        request = youtube.commentThreads().list(
            # ... params ...
            pageToken=response["nextPageToken"]
        )
    else:
        break
\end{lstlisting}

\section{User Interface Implementation}
The UI is structured using `st.tabs` to separate different analytical views. This prevents information overload. The "Recommendations" tab uses heuristic logic: if a topic cluster has a high percentage of negative sentiment, it is flagged as a "Pain Point".
