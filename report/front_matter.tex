    % Certificate
    \chapter*{Certificate}
    \addcontentsline{toc}{chapter}{Certificate}

    This is to certify that the project report entitled \textbf{``Brand Intel: AI-Powered Social Media Analytics''} submitted by \textbf{Priyanshu Kumar Sharma} in partial fulfillment of the requirements for the award of the degree of \textbf{Bachelor of Technology in Information security, specialising in Cloud Technology and Information Security} is a bonafide record of the work carried out by him under my supervision and guidance.

    The matter embodied in this report has not been submitted to any other University or Institute for the award of any degree or diploma.

    \vspace{3cm}

    \begin{minipage}[t]{0.45\textwidth}
        \textbf{Subject Faculty} \\
        \\
        Name: Ranjana Singh \\
        Designation: Professor
    \end{minipage}
    \hfill



    \newpage

    % Declaration
    \chapter*{Declaration}
    \addcontentsline{toc}{chapter}{Declaration}

    I, \textbf{Priyanshu Kumar Sharma}, hereby declare that the project report entitled \textbf{``Brand Intel: AI-Powered Social Media Analytics''} submitted for the partial fulfillment of the requirements for the degree of Bachelor of Technology is my original work and the project has not formed the basis for the award of any other degree, diploma, fellowship, or any other similar title.

    \vspace{3cm}

    \noindent
    \textbf{Place:} Pune \hfill \textbf{(Priyanshu Kumar Sharma)} \\
    \textbf{Date:} \today



    \newpage

    % Abstract
    \chapter*{Abstract}
    \addcontentsline{toc}{chapter}{Abstract}

    In the contemporary digital landscape, social media platforms have evolved into primary channels for customer feedback and brand interaction. For marketing teams, the sheer volume of unstructured textual data generated daily presents a significant challenge: manually analyzing thousands of comments to gauge public sentiment is both time-consuming and prone to human error.

    This project, \textbf{Brand Intel}, addresses this critical gap by developing an automated, AI-powered analytics dashboard. Leveraging \textbf{Natural Language Processing (NLP)} techniques, the system ingests raw social media comments—specifically from YouTube campaigns—and transforms them into actionable strategic insights.

    The core methodology involves a multi-stage pipeline: data acquisition via the YouTube Data API v3, rigorous text preprocessing, rule-based sentiment analysis using TextBlob, and unsupervised topic modeling via K-Means clustering. The results are visualized in a high-performance web application built with Streamlit, offering real-time metrics on sentiment distribution and emerging conversation themes.

    The tool demonstrates its efficacy by enabling users to input any YouTube video link and instantly generate a comprehensive analysis of the viewer comments. The system successfully identifies specific pain points, recurring themes, and sentiment trends within the discussion. These insights allow brand managers and content creators to make data-driven decisions, such as refining content strategies or addressing viewer concerns, thereby closing the loop between audience feedback and brand strategy.

    \textbf{Keywords:} Natural Language Processing, Sentiment Analysis, Topic Modeling, Social Media Analytics, Brand Strategy, Python, Streamlit.
